\documentclass[12pt,a4paper]{article}
\usepackage[czech]{babel}
\usepackage[utf8]{inputenc}
\usepackage[T1]{fontenc}
\usepackage{amsmath, amssymb}
\usepackage{graphicx}
\usepackage{fullpage}
\usepackage[a4paper, left=1.2cm, right=2.2cm, top=1.5cm, bottom=1.5cm]{geometry}

\begin{document}

\thispagestyle{empty}

\section*{Zápočtový test z NSES}

Jméno studenta: \rule{8cm}{0.4pt}

\vspace{0.5cm}

Na obrázku níže je znázorněna plně propojená neuronová síť.
V buňkách je znázorněna složka vektoru v dané vrstvě, nad propojeními jsou názvy matic
reprezentující váhy v jednotlivých vrstvách. Pro jednoduchost uvažujeme vrstvy bez prahu
(\emph{bias}) a bez aktivačních funkcí. Pro potřeby definování ztrátové funkce uvažujeme
na vstupu pouze jeden vektor.

\vspace{0.3cm}

\includegraphics[width=0.6\textwidth]{$FIGURE_FILE}

\vspace{0.3cm}

\begin{enumerate}
    \item Doplňte strukturu sítě tak, aby umožňovala řešit typ úlohy \textbf{$TASK_SHORT}.
    Uveďte vhodnou aktivační funkci / vrstvu za poslední vrstvou a zakreslete
    výsledný počet neuronů a jejich propojení s předcházející vrstvou.

    \item Definujte požadovanou \textbf{výstupní hodnotu} $$\vec{y}$$ podle typu úlohy
    (jaké hodnoty může nabývat, jaký bude její tvar). \\[1ex]
    \emph{Nápověda:} stručně popište, zda jde o skalár, vektor, one-hot vektor,
    vektor pravděpodobností apod.

    \item Zvolte vhodnou \textbf{ztrátovou funkci} pro řešení dané úlohy a zapište její vzorec.
    Použijte notaci výstupu sítě $$\vec{\hat{y}}$$ a požadované hodnoty $$\vec{y}$$. \\[0.5ex]

    \item Rozepište \textbf{dopředný průchod} sítí jako maticové operace ve správném tvaru.
    Označte vstupní vektor $$x$$, mezivýstupy v jednotlivých vrstvách a výstup sítě
    ($$\vec{s}, \vec{t}, \vec{z}, \vec{\hat{y}}$$).

    \item Napište vzorec pro výpočet \textbf{gradientu ztrátové funkce} vzhledem k prvku
    matice $$S_{$INDEX1 $INDEX2}$$. Výsledek zapište jako vhodnou aplikaci řetězového pravidla
    (\emph{chain rule}).
\end{enumerate}

\end{document}
